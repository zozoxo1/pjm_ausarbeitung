\documentclass[a4paper,12pt]{scrartcl}

\usepackage[utf8]{inputenc}
\usepackage[ngerman]{babel}
\usepackage{amsmath}
\usepackage[a4paper, left=3cm, right=2.5cm, top=2.5cm, bottom=3cm]{geometry}
\usepackage{graphicx}
\usepackage{csquotes}

\title {\textbf{Abschlussbericht}}
\author{Alexander Widera \\ Fabian Waltermann \\ Zoe Luca Günther}

\makeatletter
\def\@maketitle{%
  \newpage
  \null
  \vskip 15em%
  \begin{center}%
  \let \footnote \thanks
    {\LARGE \@title \par}%
    \vskip 1em%
    {\Large IT-Projektmanagement \\ Die Lehre im Online- und im Präsenzsemester - Was ist für das Klima besser?\par}%
    \vskip 5em%
    {\large
      \lineskip .5em%
      \begin{tabular}[t]{c}%
        \@author
      \end{tabular}\par}%
    \vskip 1em%
    {\large \@date}%
  \end{center}%
  \par
  \vskip 1.5em}
\makeatother

\begin{document}

\pagenumbering{Roman}
\maketitle
\newpage
\tableofcontents
\newpage
\pagenumbering{arabic}

\section{Vorwort}
Vor Ihnen liegt das Projekt zur Teilprüfung 1, welches in Zusammenarbeit von Zoe
Luca Günther, Fabian Waltermann und Alexander Widera erstellt wurde. Das Ziel dieses
Projektes ist .... Außerdem wird die Zusammenarbeit in der Gruppe dokumentiert.
Wir setzen uns von KW19 bis KW25 mit diesem Projekt auseinander.

\section{Unsere Fragestellung}
\subsection{Relevanz dieser Arbeit}
Diese Arbeit ist von großer Relevanz, da das Klima in der heutigen Zeit eine sehr große und wichtige Rolle spielt. Durch die Bevorstehende Öl- und Gaskriese wird dieses Thema noch mehr in den Vordergrund gedrückt, da die weitere Situation noch sehr unklar ist. Gerade deswegen sollte man darüber nachdenken, wie die weiteren Schritte für das Wintersemester aussehen könnten, um effizient die vorhandenen Ressourcen zu nutzen und so Klimafreundlich wie es geht sich zu verhalten.

\subsection{Unsere These}
Im Laufe des Berichts, werden wir die These belegen oder widerlegen. Wir können zum jetzigen Zeitpunkt also nur Vermutungen aufstellen. Wir können nicht auf jede Situation eingehen, welche momentan auftreten können, da sich diese von heute auf morgen ändern können. Dies konnten wir im Laufe unserer Recherchearbeit feststellen.\\\\

Unsere aufgestellte These:\\
\blockquote{Aus unserer Sicht ist für das Klima am besten, wenn wir weiterhin Homeoffice für das Wintersemester festlegen. Dies hat den Vorteil, dass sich viele Anfahrtswege mit dem Auto sparen lassen, und man nicht das ganze Fachhochschulgebäude beheizen muss und somit Gas gespart werden kann. Außerdem kann man den Zeitlichen Aspekt mit einbeziehen, wo wir vermuten können, dass die Zeitersparnis größer im Homeoffice als in der Präsenzlehre.}

\section{Energieverbrauch}
\subsection{Energieverbrauch beim Pendeln}
\subsubsection{ÖPNV Infrastruktur}
Wie man in unseren Eigenen Erfahrungen nachlesen kann, ist der Ausbau des ÖPNV nicht ausgereift genug, um sichere Verbindungen zu gewährleisten. Es kommt oft zu Ausfällen, welche sich nicht direkt als Kunde feststellen lassen und meist keine Alternative gegeben ist. Aus dem Infoportal von mobil.nrw, worin Infos zur Infrastruktur von Nordrhein-Westfalen geschrieben sind, geht heraus, dass der Schienennahverkehr einer sehr hohen Belastung ausgesetzt ist, dadurch dass die Strecken von Schienenpersonennah$-$, Fernverkehrs sowie des Güterverkehrs parallel genutzt werden. Aus dem SPNV$-$ Qualitätsbericht geht hervor, dass sich die Belastung während des 1. Lockdowns verringert hat, man konnte also eine erhöhte Pünktlichkeit feststellen. \\
Aus der Grafik ``Erreichbarkeit von Bus und Bahn'' der Allianz pro Schiene von 08/2021 geht hervor, dass im Bundesdurchschnitt ca 91\% der Menschen in Deutschland maximal 600 Meter von einer Bushaltestelle oder 1200 Meter von einer Bahnhaltestelle entfernt wohnen, welche mindestens 20 mal am Tag abfahren. Diese Werte werden mithilfe der Luftlinie gemessen. Dies ist somit ein großer Anteil der Bevölkerung, welche die Möglichkeit haben, einen Bus oder die Bahn zu benutzen, wenn man die finanziellen mittel außer Acht lässt. Die Zugausfälle haben sich im Jahr 2020 nach dem Lockdown jedoch mehr als verdoppelt im Vergleich zum Jahr 2019. Dies ist vor allem durch den Grundfahrplan im 1. Lockdown zu begründen, hinzu kommen jedoch weitere Äußere Einwirkungen wie wachsendes Bauvolumen und somit steigende Streckenwartungen.\\
Zur Berücksichtigung der Pünktlichkeit kann man feststellen, dass sich diese in den letzten 2 Jahren gesteigert hat (von 84.8\% im Jahr 2019 auf 86.8\% im Jahr 2020).\\
Aus dem Personenbeförderungsgesetz geht hervor, dass die Barrierefreiheit bis 2022 auf den gesamten ÖPNV ausgedehnt werden muss, was einige Baustellen an Bahnhöfen erklären kann.

\subsubsection{9€ Ticket}
\subsection{Energieverbrauch im Homeoffice}
\subsubsection{eigener Energieverbrauch}
\subsection{Gas- und Ölimport Abhängigkeit}
\subsection{Gasmangel}
\subsection{Energieverbrauch Zuhause und an der Fachhochschule}
\subsection{E-Auto vs Verbrenner}

\section{Zeitmanagement}
\subsection{Zeitmanagement im Homeoffice}
\subsection{Zeitmanagement beim Pendeln}
\subsection{Eigene Erfahrungen}
Aus eigener Erfahrung können wir sagen, dass Homeoffice die effizienteste Methode ist, wenn es um das Zeitmanagement geht. Alle Teammitglieder von uns haben einen Anfahrtsweg zur Fachhochschule, welcher länger als eine Stunde ist. Dazu kommt noch die Teils schlechte ÖPNV Anbindung und die dazwischenliegenden Baustellen. Mit dem Auto lässt sich etwas Zeit sparen, jedoch sind die Autobahnanbindungen teils gar nicht gegeben und teils sehr stark befahren, weshalb man oft in einen Stau gerät. Wir konnten im Laufe des Semesters in unserem Team feststellen, dass das lernen im Homeoffice effizienter Funktioniert hat, dadurch dass man keine Zwangspausen halten muss (zum Beispiel die 30 Minuten Maskenpause) und man sich seine Zeit selber einteilen kann, was das eigenständige Lernen fördert. Wenn wir Terminlich beschränkt waren, war es dank der Online Vorlesung möglich, daran teilzunehmen, was in der Präsenzlehre durch den eben genannten Anfahrtsweg nicht möglich wäre. Da zwischen unseren Teammitgliedern ein zu großer Abstand liegt, ist es nicht möglich eine Fahrgemeinschaft zu bilden, um die Umwelt weniger zu belasten.

\section{Zusammenfassung}
\section{Fazit}

\section{Feedback der Gruppenmitglieder}
\subsection{Alexander Widera}
\subsection{Fabian Waltermann}
\subsection{Zoe Günther}

\newpage
\section{Quellenverzeichnis}
\begin{itemize}
\item https://www.geeksforgeeks.org/space-optimized-solution-lcs/ \\ Stand: 01.06.2022
\item https://www.geeksforgeeks.org/python-program-for-longest-common-subsequence/ \\ Stand: 01.06.2022
\item Titel: Algorithmen - Eine Einführung \\ Autoren: Thomas H. Cormen, Charles E. Leiserson, Ronald Rivest, Clifford Stein \\ Erscheinungsjahr: 2013 \\ Auflage: 3. Auflage \\ ISBN: 978-3-486-74861-1
\item https://www.techiedelight.com/de/longest-common-subsequence-finding-lcs/ \\ Stand: 26.05.2022
\end{itemize}

\end{document}