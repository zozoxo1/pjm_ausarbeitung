\documentclass[a4paper,12pt]{scrartcl}

\usepackage[utf8]{inputenc}
\usepackage[ngerman]{babel}
\usepackage{amsmath}
\usepackage[a4paper, left=3cm, right=2.5cm, top=2.5cm, bottom=3cm]{geometry}
\usepackage{graphicx}
\usepackage{csquotes}

\title {\textbf{Abschlussbericht}}
\author{Alexander Widera \\ Fabian Waltermann \\ Zoe Luca Günther}

\makeatletter
\def\@maketitle{%
  \newpage
  \null
  \vskip 15em%
  \begin{center}%
  \let \footnote \thanks
    {\LARGE \@title \par}%
    \vskip 1em%
    {\Large IT-Projektmanagement \\ Die Lehre im Online- und im Präsenzsemester - Was ist für das Klima besser?\par}%
    \vskip 5em%
    {\large
      \lineskip .5em%
      \begin{tabular}[t]{c}%
        \@author
      \end{tabular}\par}%
    \vskip 1em%
    {\large \@date}%
  \end{center}%
  \par
  \vskip 1.5em}
\makeatother

\begin{document}

\pagenumbering{Roman}
\maketitle
\newpage
\tableofcontents
\newpage
\pagenumbering{arabic}

\section{Vorwort}
Vor Ihnen liegt das Projekt zur Teilprüfung 1, welches in Zusammenarbeit von Zoe
Luca Günther, Fabian Waltermann und Alexander Widera erstellt wurde. Das Ziel dieses
Projektes ist .... Außerdem wird die Zusammenarbeit in der Gruppe dokumentiert.
Wir setzen uns von KW19 bis KW25 mit diesem Projekt auseinander.

\section{Unsere Fragestellung}
\subsection{Relevanz dieser Arbeit}
Diese Arbeit ist von großer Relevanz, da das Klima in der heutigen Zeit eine sehr große und wichtige Rolle spielt. Durch die Bevorstehende Öl- und Gaskriese wird dieses Thema noch mehr in den Vordergrund gedrückt, da die weitere Situation noch sehr unklar ist. Gerade deswegen sollte man darüber nachdenken, wie die weiteren Schritte für das Wintersemester aussehen könnten, um effizient die vorhandenen Ressourcen zu nutzen und so Klimafreundlich wie es geht sich zu verhalten.

\subsection{Unsere These}
Im Laufe des Berichts, werden wir die These belegen oder widerlegen. Wir können zum jetzigen Zeitpunkt also nur Vermutungen aufstellen. Wir können nicht auf jede Situation eingehen, welche momentan auftreten können, da sich diese von heute auf morgen ändern können. Dies konnten wir im Laufe unserer Recherchearbeit feststellen.\\\\

Unsere aufgestellte These:\\
\blockquote{Aus unserer Sicht ist für das Klima am besten, wenn wir weiterhin Homeoffice für das Wintersemester festlegen. Dies hat den Vorteil, dass sich viele Anfahrtswege mit dem Auto sparen lassen, und man nicht das ganze Fachhochschulgebäude beheizen muss und somit Gas gespart werden kann. Außerdem kann man den Zeitlichen Aspekt mit einbeziehen, wo wir vermuten können, dass die Zeitersparnis größer im Homeoffice als in der Präsenzlehre.}

\section{Energieverbrauch}
\subsection{Energieverbrauch beim Pendeln}
\subsubsection{ÖPNV Infrastruktur}
Wie man in unseren Eigenen Erfahrungen nachlesen kann, ist der Ausbau des ÖPNV nicht ausgereift genug, um sichere Verbindungen zu gewährleisten. Es kommt oft zu Ausfällen, welche sich nicht direkt als Kunde feststellen lassen und meist keine Alternative gegeben ist. Aus dem Infoportal von mobil.nrw, worin Infos zur Infrastruktur von Nordrhein-Westfalen geschrieben sind, geht heraus, dass der Schienennahverkehr einer sehr hohen Belastung ausgesetzt ist, dadurch dass die Strecken von Schienenpersonennah$-$, Fernverkehrs sowie des Güterverkehrs parallel genutzt werden. Aus dem SPNV$-$ Qualitätsbericht geht hervor, dass sich die Belastung während des 1. Lockdowns verringert hat, man konnte also eine erhöhte Pünktlichkeit feststellen. \\
Aus der Grafik ``Erreichbarkeit von Bus und Bahn'' der Allianz pro Schiene von 08/2021 geht hervor, dass im Bundesdurchschnitt ca 91\% der Menschen in Deutschland maximal 600 Meter von einer Bushaltestelle oder 1200 Meter von einer Bahnhaltestelle entfernt wohnen, welche mindestens 20 mal am Tag abfahren. Diese Werte werden mithilfe der Luftlinie gemessen. Dies ist somit ein großer Anteil der Bevölkerung, welche die Möglichkeit haben, einen Bus oder die Bahn zu benutzen, wenn man die finanziellen mittel außer Acht lässt. Die Zugausfälle haben sich im Jahr 2020 nach dem Lockdown jedoch mehr als verdoppelt im Vergleich zum Jahr 2019. Dies ist vor allem durch den Grundfahrplan im 1. Lockdown zu begründen, hinzu kommen jedoch weitere Äußere Einwirkungen wie wachsendes Bauvolumen und somit steigende Streckenwartungen.\\
Zur Berücksichtigung der Pünktlichkeit kann man feststellen, dass sich diese in den letzten 2 Jahren gesteigert hat (von 84.8\% im Jahr 2019 auf 86.8\% im Jahr 2020).\\
Aus dem Personenbeförderungsgesetz geht hervor, dass die Barrierefreiheit bis 2022 auf den gesamten ÖPNV ausgedehnt werden muss, was einige Baustellen an Bahnhöfen erklären kann.

\subsubsection{9€ Ticket}
\subsubsection{Vergleich Energieverbrauch ÖPNV und PKW}
\subsection{Energieverbrauch im Homeoffice}
\subsubsection{eigener Energieverbrauch}
\subsection{Energieverbrauch an der Fachhochschule}
\subsection{Gas- und Ölimport Abhängigkeit}
\subsection{Gasmangel}
\subsection{E-Auto vs Verbrenner}

\section{Zeitmanagement}
In diesem Abschnitt behandeln wir das Zeitmanagement, wie es aus unsere Sicht am effizientesten und am objektivsten gehandhabt werden kann.

\subsection{Zeitmanagement im Homeoffice}
Ein typischer Ablauf im Homeoffice, welcher besonders effizient sein kann, startet morgens zur ersten Vorlesung, wobei es reicht, wenn man dabei 15 Minuten vor der Vorlesung aufsteht. Nebenbei kann man sich etwas zu essen machen, da die Küche direkt nebenan ist. Einen Fahrtweg kann man hierbei ausschließen. Nachdem die Vorlesung vorbei ist und eine Pause eingeplant ist, kann man sich diese beliebig einteilen. Man könnte direkt den Stoff der nächsten Vorlesung vorbereiten, oder den aus der vorherigen nachbereiten. Hierbei kann man sich mit seinen Kommiliton*innen online verabreden, um im Team gemeinsam zu arbeiten, da es in der Gruppe teils besser zu lernen ist. Sollten zwischendurch private Termine anstehen, lassen sich diese effizient in den Tagesplan einbringen, da man zuhause nicht sonderlich gebunden ist. Nach dem Ende der letzten Vorlesung des Tages, wird keine Zeit benötigt, um den Weg nach hause anzutreten, es kann direkt weiterer Stoff bearbeitet werden oder sofort eine Pause eingelegt werden, in der man Essen kann, oder sonstigen Verpflichtungen nachkommen kann.

\subsection{Zeitmanagement beim Pendeln}
Ein typischer Ablauf beim Pendeln, welcher besonders effizient sein kann, lässt vermuten, dass die meisten Student*innen nicht aus Iserlohn oder der unmittelbaren Umgebung kommen, sondern einen weiten Anfahrtsweg auf sich nehmen müssen. Dies wird üblicherweise mit dem ÖPNV oder dem PKW bewältigt. Somit muss hier morgens in den vermutlich meisten Fällen einige Stunden vor Beginn der ersten Vorlesung sich vorbereitet werden, auf den bevorstehenden Weg. Nach der Vorlesung kann man sich mit seinen Kommiliton*innen getroffen werden, da zu Corona Zeiten eine zwangs Maskenpause eingelegt werden muss (30 Minuten). In dieser Zeit muss man das Gebäude verlassen, da sonst die Coronaschutzverordnung nicht eingehalten werden kann. Die einzige Möglichkeit, sich innerhalb des Gebäudes zu treffen, ist in den Pausen, jedoch mit Maske oder in der Mensa (Stand: 14.04.2022). Nach der letzten Vorlesung fahren die meisten nach Hause, um seinen täglichen Aufgaben nachzukommen, oder Terminen nachzugehen. Wie hier schon auffällt, lassen sich diese Dinge nicht zwischen die Vorlesungen unterbringen, da vermutlich der Weg zu weit ist. Sollte ein Termin anstehen, muss mehr Zeit eingeplant werden, um zum Termin zu kommen. Dies kann dafür sorgen, dass Vorlesungen ausgelassen werden müssen und somit Unterrichtsstoff verpasst wird und zu einem anderen Zeitpunkt nachgeholt werden sollte.

\subsection{Eigene Erfahrungen}
Aus eigener Erfahrung können wir sagen, dass Homeoffice die effizienteste Methode ist, wenn es um das Zeitmanagement geht. Alle Teammitglieder von uns haben einen Anfahrtsweg zur Fachhochschule, welcher länger als eine Stunde ist. Dazu kommt noch die Teils schlechte ÖPNV Anbindung und die dazwischenliegenden Baustellen. Mit dem Auto lässt sich etwas Zeit sparen, jedoch sind die Autobahnanbindungen teils gar nicht gegeben und teils sehr stark befahren, weshalb man oft in einen Stau gerät. Wir konnten im Laufe des Semesters in unserem Team feststellen, dass das lernen im Homeoffice effizienter Funktioniert hat, dadurch dass man keine Zwangspausen halten muss (zum Beispiel die 30 Minuten Maskenpause) und man sich seine Zeit selber einteilen kann, was das eigenständige Lernen fördert. Wenn wir Terminlich beschränkt waren, war es dank der Online Vorlesung möglich, daran teilzunehmen, was in der Präsenzlehre durch den eben genannten Anfahrtsweg nicht möglich wäre. Da zwischen unseren Teammitgliedern ein zu großer Abstand liegt, ist es nicht möglich eine Fahrgemeinschaft zu bilden, um die Umwelt weniger zu belasten.

\section{Zusammenfassung}
\section{Fazit}

\section{Feedback der Gruppenmitglieder}
\subsection{Alexander Widera}
Aus meiner Sicht lief das Projekt sehr gut. Alle restlichen Teammitglieder hielten die aufgestellte Arbeitsaufteilung ein. Leider gab es Probleme mit der Zusammenarbeit, da eines der Team Mitglieder sich spontan entschied das Studium nicht länger fortzuführen. Dennoch, trotz der unerwarteten Probleme konnte schnell die Aufgabenverteilung geändert werden und das Projekt, dennoch gut da die Zusammenarbeit der restlichen Gruppenmitgliedern gut geklappt hat. Alles in einem kann ich sagen, dass es trotz der Probleme gut geklappt hat und ich denke das dieses Projekt mir weiter geholfen hat meine Projekte besser zu strukturieren.

\subsection{Fabian Waltermann}
\subsection{Zoe Günther}
Das Projekt war sehr spannend und die Planung lief reibungslos ab. Es gab innerhalb der Gruppe jedoch ein Problem, welches sich schnell lösen ließ. Ein Gruppenmitglied hat sich kurzfristig dazu entschlossen, dass Studium nicht weiterzuführen. Diese Aufgaben musste also auf die restlichen Teammitglieder verteilt werden. Nachdem das geklärt war, lief aus auch schon wieder reibungslos ab. Der Austausch in der Gruppe lief über vorher definierte Kanäle und die Quellen wurden jedem Teammitglied zur Verfügung gestellt.
Zusammengefasst kann ich sagen, dass ich viel in diesem Projekt lernen konnte, was das Managen von größeren Projekten betrifft und wie man handeln kann, wenn ein Mitglied mal ausfällt, wie es in der Realität auch mal passieren kann. Ich hoffe, dass ich diese Erfahrungen in ein anderes Projekt übernehmen kann.

\newpage
\section{Quellenverzeichnis}
\begin{itemize}
\item https://www.staedtetag.de/publikationen/weitere-publikationen/barrierefreiheit-oepnv-2014 \\ Stand: 09.05.2022
\item https://www.allianz-pro-schiene.de/themen/dossiers/erreichbarkeitsranking/ \\ Stand: 11.05.2022
\item https://infoportal.mobil.nrw/information-service/spnv-qualitaetsbericht.html \\ Stand: 12.05.2022
\item https://infoportal.mobil.nrw/organisation-finanzierung/infrastruktur.html \\ Stand: 12.05.2022
\end{itemize}

\end{document}